This is a test file with some referenced equations, line \[ this \]

Some equations might be inside of text, \[ for example \] this one.

Some equations might be on start of paragraphs:

\[ start \] of paragraph.

Other might be on separate paragraphs of their own:

\[ separate \]

Some of those can be labelled:

This is a test file with some referenced equations, line
\begin{equation} this \label{eq:0}\end{equation}

Some equations might be inside of text,
\begin{equation} for example \label{eq:1}\end{equation} this one.

Some equations might be on start of paragraphs:

\begin{equation} start \label{eq:2}\end{equation} of paragraph.

Other might be on separate paragraphs of their own:

\begin{equation} separate \label{eq:3}\end{equation}

Then they can be referenced:

Individually eq.~\ref{eq:0}, eq.~\ref{eq:1}, eq.~\ref{eq:2},
eq.~\ref{eq:3}

Or in groups eqns.~\ref{eq:0}, \ref{eq:1}, \ref{eq:3}

Groups will be compacted
eqns.~\ref{eq:0}, \ref{eq:1}, \ref{eq:3}, \ref{eq:2}

Unknown references will print labels
eqns.~\ref{eq:0}, \ref{eq:none}, \ref{eq:3}, \ref{eq:2}

Reference prefix will override default prefix Equation \ref{eq:0},
eqns.~\ref{eq:3}, \ref{eq:2}

References with \texttt{-} prepended won't have prefix at all:
\ref{eq:0}, \ref{eq:1}, eqns.~\ref{eq:2}, \ref{eq:3}

References with suffix will have suffix printed after index
(configurable): eqns.~\ref{eq:0}, \ref{eq:1}, \ref{eq:2}
